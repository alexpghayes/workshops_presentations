\documentclass[12pt,ignorenonframetext,compress]{beamer}
\setbeamertemplate{caption}[numbered]
\setbeamertemplate{caption label separator}{: }
\setbeamercolor{caption name}{fg=normal text.fg}
\beamertemplatenavigationsymbolsempty
\usepackage{lmodern}
\usepackage{amssymb,amsmath}
\usepackage{ifxetex,ifluatex}
\usepackage{fixltx2e} % provides \textsubscript
\ifnum 0\ifxetex 1\fi\ifluatex 1\fi=0 % if pdftex
  \usepackage[T1]{fontenc}
  \usepackage[utf8]{inputenc}
\else % if luatex or xelatex
  \ifxetex
    \usepackage{mathspec}
  \else
    \usepackage{fontspec}
  \fi
  \defaultfontfeatures{Ligatures=TeX,Scale=MatchLowercase}
\fi
\usetheme[]{metropolis}
% use upquote if available, for straight quotes in verbatim environments
\IfFileExists{upquote.sty}{\usepackage{upquote}}{}
% use microtype if available
\IfFileExists{microtype.sty}{%
\usepackage{microtype}
\UseMicrotypeSet[protrusion]{basicmath} % disable protrusion for tt fonts
}{}
\newif\ifbibliography
\hypersetup{
            pdftitle={Your First R Package},
            pdfauthor={Alex Hayes},
            pdfborder={0 0 0},
            breaklinks=true}
\urlstyle{same}  % don't use monospace font for urls
\usepackage{color}
\usepackage{fancyvrb}
\newcommand{\VerbBar}{|}
\newcommand{\VERB}{\Verb[commandchars=\\\{\}]}
\DefineVerbatimEnvironment{Highlighting}{Verbatim}{commandchars=\\\{\}}
% Add ',fontsize=\small' for more characters per line
\usepackage{framed}
\definecolor{shadecolor}{RGB}{248,248,248}
\newenvironment{Shaded}{\begin{snugshade}}{\end{snugshade}}
\newcommand{\KeywordTok}[1]{\textcolor[rgb]{0.13,0.29,0.53}{\textbf{#1}}}
\newcommand{\DataTypeTok}[1]{\textcolor[rgb]{0.13,0.29,0.53}{#1}}
\newcommand{\DecValTok}[1]{\textcolor[rgb]{0.00,0.00,0.81}{#1}}
\newcommand{\BaseNTok}[1]{\textcolor[rgb]{0.00,0.00,0.81}{#1}}
\newcommand{\FloatTok}[1]{\textcolor[rgb]{0.00,0.00,0.81}{#1}}
\newcommand{\ConstantTok}[1]{\textcolor[rgb]{0.00,0.00,0.00}{#1}}
\newcommand{\CharTok}[1]{\textcolor[rgb]{0.31,0.60,0.02}{#1}}
\newcommand{\SpecialCharTok}[1]{\textcolor[rgb]{0.00,0.00,0.00}{#1}}
\newcommand{\StringTok}[1]{\textcolor[rgb]{0.31,0.60,0.02}{#1}}
\newcommand{\VerbatimStringTok}[1]{\textcolor[rgb]{0.31,0.60,0.02}{#1}}
\newcommand{\SpecialStringTok}[1]{\textcolor[rgb]{0.31,0.60,0.02}{#1}}
\newcommand{\ImportTok}[1]{#1}
\newcommand{\CommentTok}[1]{\textcolor[rgb]{0.56,0.35,0.01}{\textit{#1}}}
\newcommand{\DocumentationTok}[1]{\textcolor[rgb]{0.56,0.35,0.01}{\textbf{\textit{#1}}}}
\newcommand{\AnnotationTok}[1]{\textcolor[rgb]{0.56,0.35,0.01}{\textbf{\textit{#1}}}}
\newcommand{\CommentVarTok}[1]{\textcolor[rgb]{0.56,0.35,0.01}{\textbf{\textit{#1}}}}
\newcommand{\OtherTok}[1]{\textcolor[rgb]{0.56,0.35,0.01}{#1}}
\newcommand{\FunctionTok}[1]{\textcolor[rgb]{0.00,0.00,0.00}{#1}}
\newcommand{\VariableTok}[1]{\textcolor[rgb]{0.00,0.00,0.00}{#1}}
\newcommand{\ControlFlowTok}[1]{\textcolor[rgb]{0.13,0.29,0.53}{\textbf{#1}}}
\newcommand{\OperatorTok}[1]{\textcolor[rgb]{0.81,0.36,0.00}{\textbf{#1}}}
\newcommand{\BuiltInTok}[1]{#1}
\newcommand{\ExtensionTok}[1]{#1}
\newcommand{\PreprocessorTok}[1]{\textcolor[rgb]{0.56,0.35,0.01}{\textit{#1}}}
\newcommand{\AttributeTok}[1]{\textcolor[rgb]{0.77,0.63,0.00}{#1}}
\newcommand{\RegionMarkerTok}[1]{#1}
\newcommand{\InformationTok}[1]{\textcolor[rgb]{0.56,0.35,0.01}{\textbf{\textit{#1}}}}
\newcommand{\WarningTok}[1]{\textcolor[rgb]{0.56,0.35,0.01}{\textbf{\textit{#1}}}}
\newcommand{\AlertTok}[1]{\textcolor[rgb]{0.94,0.16,0.16}{#1}}
\newcommand{\ErrorTok}[1]{\textcolor[rgb]{0.64,0.00,0.00}{\textbf{#1}}}
\newcommand{\NormalTok}[1]{#1}

% Prevent slide breaks in the middle of a paragraph:
\widowpenalties 1 10000
\raggedbottom

\AtBeginPart{
  \let\insertpartnumber\relax
  \let\partname\relax
  \frame{\partpage}
}
\AtBeginSection{
  \ifbibliography
  \else
    \let\insertsectionnumber\relax
    \let\sectionname\relax
    \frame{\sectionpage}
  \fi
}
\AtBeginSubsection{
  \let\insertsubsectionnumber\relax
  \let\subsectionname\relax
  \frame{\subsectionpage}
}

\setlength{\parindent}{0pt}
\setlength{\parskip}{6pt plus 2pt minus 1pt}
\setlength{\emergencystretch}{3em}  % prevent overfull lines
\providecommand{\tightlist}{%
  \setlength{\itemsep}{0pt}\setlength{\parskip}{0pt}}
\setcounter{secnumdepth}{0}
\setbeamercolor{frametitle}{bg=gray}
\hypersetup{colorlinks,citecolor=orange,filecolor=red,linkcolor=brown,urlcolor=blue}

\title{Your First R Package}
\subtitle{Download these slides from goo.gl/aiuNFp}
\author{Alex Hayes}
\date{2018-02-20}

\begin{document}
\frame{\titlepage}

\begin{frame}[fragile]{Pre-requisites}

\begin{enumerate}
\def\labelenumi{\arabic{enumi}.}
\item
  Install RStudio + R
\item
  Install some important packages:
\end{enumerate}

\begin{Shaded}
\begin{Highlighting}[]
\NormalTok{pkgs <-}\StringTok{ }\KeywordTok{c}\NormalTok{(}\StringTok{"devtools"}\NormalTok{, }\StringTok{"usethis"}\NormalTok{,}
          \StringTok{"roxygen2"}\NormalTok{, }\StringTok{"testthat"}\NormalTok{,}
          \StringTok{"ggplot2"}\NormalTok{, }\StringTok{"meme"}\NormalTok{)}
\KeywordTok{install.packages}\NormalTok{(pkgs)}
\end{Highlighting}
\end{Shaded}

\end{frame}

\begin{frame}{Why Should You Use R Packages}

\begin{itemize}
\tightlist
\item
  Keep code that you use frequently in one place:
  \href{https://github.com/alexpghayes/hayeslib}{hayeslib}
\item
  Fewer copy-paste errors
\item
  Easy to share code with others
\item
  Understand why packages work the way they do
\item
  Learn where documentation lives
\item
  Nerd cred
\end{itemize}

\end{frame}

\begin{frame}[fragile]{Documentation pop-quiz}

\begin{enumerate}
\def\labelenumi{\arabic{enumi}.}
\tightlist
\item
  How you find the documentation for the \texttt{lm} function?
\item
  How do you see the source code for the \texttt{lm} function?
\end{enumerate}

Take 2 minutes

\end{frame}

\begin{frame}{Our Goal Today}

Create a personal package with two functions

\end{frame}

\begin{frame}[fragile]{Function 1: Get OLS coefficients}

\begin{Shaded}
\begin{Highlighting}[]
\NormalTok{ols_coefs <-}\StringTok{ }\ControlFlowTok{function}\NormalTok{(X, y) \{}
  \KeywordTok{solve}\NormalTok{(}\KeywordTok{t}\NormalTok{(X) }\OperatorTok\StringTok{ }\NormalTok{X) }\OperatorTok\StringTok{ }\KeywordTok{t}\NormalTok{(X) }\OperatorTok\StringTok{ }\NormalTok{y}
\NormalTok{\}}
\end{Highlighting}
\end{Shaded}

\end{frame}

\begin{frame}[fragile]{Function 2: MEMES}

\begin{Shaded}
\begin{Highlighting}[]
\NormalTok{ernst_meme <-}\StringTok{ }\ControlFlowTok{function}\NormalTok{(}\DataTypeTok{upper =} \StringTok{""}\NormalTok{, }\DataTypeTok{lower =} \StringTok{""}\NormalTok{,}
                       \DataTypeTok{vjust =} \FloatTok{0.25}\NormalTok{, ...) \{}
  \ControlFlowTok{if}\NormalTok{ (.Platform}\OperatorTok{$}\NormalTok{OS.type }\OperatorTok{==}\StringTok{ "windows"}\NormalTok{) \{}
    \KeywordTok{windowsFonts}\NormalTok{(}\DataTypeTok{Impact =} \KeywordTok{windowsFont}\NormalTok{(}\StringTok{"Impact"}\NormalTok{),}
                 \DataTypeTok{Courier =} \KeywordTok{windowsFont}\NormalTok{(}\StringTok{"Courier"}\NormalTok{))}
\NormalTok{  \}}
\NormalTok{  u <-}\StringTok{ }\KeywordTok{system.file}\NormalTok{(}\StringTok{"extdata"}\NormalTok{, }\StringTok{"ernst.jpg"}\NormalTok{,}
                   \DataTypeTok{package =} \StringTok{"mypkg"}\NormalTok{)}
\NormalTok{  meme}\OperatorTok{::}\KeywordTok{meme}\NormalTok{(u, }\DataTypeTok{upper =}\NormalTok{ upper, }\DataTypeTok{lower =}\NormalTok{ lower,}
             \DataTypeTok{vjust =}\NormalTok{ vjust, ...)}
\NormalTok{\}}
\end{Highlighting}
\end{Shaded}

\end{frame}

\begin{frame}{Putting these functions into an R package}

Live demo: create an R package skeleton with RStudio

\end{frame}

\begin{frame}[fragile]{What are these files?}

\texttt{.gitignore}: Makes using \texttt{git} nice.

\texttt{.Rbuildignore}: You can ignore this for now.

\texttt{DESCRIPTION}: This is where all the meta-data about your package
goes. More in a slide.

\texttt{mypkg.Rproj}: Turns the directory into an RStudio project and
allows you to save RStudio settings specific to the package.

\texttt{NAMESPACE}: Controls which functions your package shows
(``exports'') to users, and which functions it depends on (``imports'').
You can ignore this file since \texttt{devtools} will create it for you.

\texttt{R}: A folder where your R code goes.

\end{frame}

\begin{frame}[fragile]{DESCRIPTION (template)}

We fill this in with the relevant info:

\begin{verbatim}
Package: mypkg
Title: What the Package Does (one line, title case)
Version: 0.0.0.9000
Authors@R: person("First", "Last",
                  email = "first.last@example.com",
                  role = c("aut", "cre"))
Description: What the package does (one paragraph).
Depends: R (>= 3.4.1)
License: What license is it under?
Encoding: UTF-8
LazyData: true
\end{verbatim}

\end{frame}

\begin{frame}[fragile]{DESCRIPTION (template filled in)}

\begin{verbatim}
Package: mypkg
Title: Calculate OLS Coefficients and Make Memes
Version: 0.0.0.9000
Authors@R: person("Alex", "Hayes",
                  email = "aph3@rice.edu",
                  role = c("aut", "cre"))
Description: Provides function to calculate OLS
    coefficients and make memes of
    professors in the Rice statistics dept.
Depends: R (>= 3.4.1)
License: MIT
Encoding: UTF-8
LazyData: true
\end{verbatim}

\end{frame}

\begin{frame}[fragile]{Putting \texttt{ols\_coefs} into our package}

\begin{Shaded}
\begin{Highlighting}[]
\CommentTok{#' Get estimates of OLS coefficients}
\CommentTok{#'}
\CommentTok{#' @param X A data matrix.}
\CommentTok{#' @param y A response vector.}
\CommentTok{#'}
\CommentTok{#' @return A vector of coefficients}
\CommentTok{#' @export}

\NormalTok{ols_coefs <-}\StringTok{ }\ControlFlowTok{function}\NormalTok{(X, y) \{}
  \KeywordTok{solve}\NormalTok{(}\KeywordTok{t}\NormalTok{(X) }\OperatorTok\StringTok{ }\NormalTok{X) }\OperatorTok\StringTok{ }\KeywordTok{t}\NormalTok{(X) }\OperatorTok\StringTok{ }\NormalTok{y}
\NormalTok{\}}
\end{Highlighting}
\end{Shaded}

Live Demo

\end{frame}

\begin{frame}[fragile]{Putting \texttt{ernst\_meme} into our package}

\begin{enumerate}
\def\labelenumi{\arabic{enumi}.}
\item
  Download
  \href{https://drive.google.com/file/d/1E9PpcxJROCTkF6rJUigAOG5gmyVH7l9c/view?usp=sharing}{this
  picture} of Ernst.
\item
  Put it in \texttt{inst/extdata/ernst.jpg}
\item
  Add a dependency on the \texttt{meme} package
\end{enumerate}

\begin{Shaded}
\begin{Highlighting}[]
\NormalTok{usethis}\OperatorTok{::}\KeywordTok{use_package}\NormalTok{(}\StringTok{"meme"}\NormalTok{)}
\end{Highlighting}
\end{Shaded}

\end{frame}

\begin{frame}[fragile]{\texttt{ernst\_meme} documentation}

\begin{Shaded}
\begin{Highlighting}[]
\CommentTok{#' Create a meme of professor Ernst}
\CommentTok{#'}
\CommentTok{#' @param upper Text to display at top of image.}
\CommentTok{#' @param lower Text to display at bottom of image.}
\CommentTok{#' @param vjust Vertical adjustment. Higher number}
\CommentTok{#'   means text closer to center of image.}
\CommentTok{#' @param ... Other arguments passed to `meme::meme`}
\CommentTok{#'}
\CommentTok{#' @return ggplot2 meme object}
\CommentTok{#' @export}
\NormalTok{ernst_meme <-}\StringTok{ }\ControlFlowTok{function}\NormalTok{(}\DataTypeTok{upper =} \StringTok{""}\NormalTok{, }\DataTypeTok{lower =} \StringTok{""}\NormalTok{,}
                       \DataTypeTok{vjust =} \FloatTok{0.25}\NormalTok{, ...) \{}
  \ControlFlowTok{if}\NormalTok{ (.Platform}\OperatorTok{$}\NormalTok{OS.type }\OperatorTok{==}\StringTok{ "windows"}\NormalTok{) \{}
    \KeywordTok{windowsFonts}\NormalTok{(}\DataTypeTok{Impact =} \KeywordTok{windowsFont}\NormalTok{(}\StringTok{"Impact"}\NormalTok{),}
                 \DataTypeTok{Courier =} \KeywordTok{windowsFont}\NormalTok{(}\StringTok{"Courier"}\NormalTok{))}
\NormalTok{  \}}
\NormalTok{  u <-}\StringTok{ }\KeywordTok{system.file}\NormalTok{(}\StringTok{"extdata"}\NormalTok{, }\StringTok{"ernst.jpg"}\NormalTok{,}
                   \DataTypeTok{package =} \StringTok{"mypkg"}\NormalTok{)}
\NormalTok{  meme}\OperatorTok{::}\KeywordTok{meme}\NormalTok{(u, }\DataTypeTok{upper =}\NormalTok{ upper, }\DataTypeTok{lower =}\NormalTok{ lower,}
             \DataTypeTok{vjust =}\NormalTok{ vjust, ...)}
\NormalTok{\}}
\end{Highlighting}
\end{Shaded}

\end{frame}

\begin{frame}{DOCUMENT! DOCUMENT! DOCUMENT!!!}

Why:

\begin{enumerate}
\def\labelenumi{\arabic{enumi}.}
\tightlist
\item
  Need to tell R to export our functions.
\item
  Need to describe what the functions do for when we inevitably forget
  in two days.
\end{enumerate}

Live demo adding in a third function from scratch!

\end{frame}

\begin{frame}{Does it work??}

\begin{enumerate}
\def\labelenumi{\arabic{enumi}.}
\tightlist
\item
  Compile the documentation!
\item
  Build and install!
\end{enumerate}

Live demo

\end{frame}

\begin{frame}[fragile]{Code from live demo}

\begin{Shaded}
\begin{Highlighting}[]
\KeywordTok{library}\NormalTok{(mypkg)}

\NormalTok{X <-}\StringTok{ }\KeywordTok{model.matrix}\NormalTok{(mpg }\OperatorTok{~}\StringTok{ }\NormalTok{hp, mtcars)}
\NormalTok{y <-}\StringTok{ }\NormalTok{mtcars}\OperatorTok{$}\NormalTok{mpg}
\KeywordTok{ols_coefs}\NormalTok{(X, y)}

\KeywordTok{ernst_meme}\NormalTok{(}
  \DataTypeTok{lower =} \StringTok{"something something probability related"}
\NormalTok{)}

\NormalTok{?ols_coefs}
\NormalTok{?ernst_meme}
\end{Highlighting}
\end{Shaded}

\end{frame}

\begin{frame}{Congrats!}

\textbf{You've just created your first R package!}

\end{frame}

\begin{frame}[fragile]{Workflow: refresher}

\begin{enumerate}
\def\labelenumi{\arabic{enumi}.}
\tightlist
\item
  Put functions into \texttt{my\_functions.R} files.
\item
  Documentation the functions in \texttt{my\_functions.R}.
\item
  Compile the documentation.
\item
  Build and install the package. (or load the functions!)
\item
  Test that things work like you'd expect.
\end{enumerate}

\end{frame}

\begin{frame}[fragile]{Loading vs building a package}

\textbf{Loading}: makes the package functions available in the current
session

\textbf{Building and installing}: Installs the package on your computer,
after which you can access the functions with \texttt{library(mypkg)}

\end{frame}

\begin{frame}[fragile]{Add a simple test}

\begin{enumerate}
\def\labelenumi{\arabic{enumi}.}
\tightlist
\item
  Start with:
\end{enumerate}

\begin{Shaded}
\begin{Highlighting}[]
\NormalTok{usethis}\OperatorTok{::}\KeywordTok{use_testthat}\NormalTok{()}
\NormalTok{usethis}\OperatorTok{::}\KeywordTok{use_test}\NormalTok{(}\StringTok{"ols"}\NormalTok{)}
\end{Highlighting}
\end{Shaded}

\end{frame}

\begin{frame}[fragile]{Testing continued}

\begin{enumerate}
\def\labelenumi{\arabic{enumi}.}
\setcounter{enumi}{1}
\tightlist
\item
  Change \texttt{tests/testthat/test-ols.R} to
\end{enumerate}

\begin{Shaded}
\begin{Highlighting}[]
\KeywordTok{context}\NormalTok{(}\StringTok{"test-ols.R"}\NormalTok{)}

\KeywordTok{test_that}\NormalTok{(}\StringTok{"multiplication works"}\NormalTok{, \{}

\NormalTok{  X <-}\StringTok{ }\KeywordTok{model.matrix}\NormalTok{(mpg }\OperatorTok{~}\StringTok{ }\NormalTok{hp, mtcars)}
\NormalTok{  y <-}\StringTok{ }\NormalTok{mtcars}\OperatorTok{$}\NormalTok{mpg}

\NormalTok{  result <-}\StringTok{ }\KeywordTok{ols_coefs}\NormalTok{(X, y)}
\NormalTok{  expected <-}\StringTok{ }\KeywordTok{coef}\NormalTok{(}\KeywordTok{lm}\NormalTok{(mpg }\OperatorTok{~}\StringTok{ }\NormalTok{hp, mtcars))}

  \KeywordTok{expect_equivalent}\NormalTok{(result, expected)}
\NormalTok{\})}
\end{Highlighting}
\end{Shaded}

\end{frame}

\begin{frame}{Even more testing}

\begin{enumerate}
\def\labelenumi{\arabic{enumi}.}
\setcounter{enumi}{2}
\tightlist
\item
  Run the tests
\item
  Change your functions until the tests pass
\end{enumerate}

\end{frame}

\begin{frame}[fragile]{Next Steps (details in Hadley's book)}

\begin{itemize}
\tightlist
\item
  Learn the RStudio keyboard shortcuts
\item
  Run the CRAN checks with \texttt{devtools::check()}
\item
  Sharing your package on Github + CRAN (add a LICENSE!)
\item
  Make a website for your package with \texttt{pkgdown}
\item
  Advertise your package
\item
  Provide vignettes (examples) showing how to use your package
\end{itemize}

\end{frame}

\begin{frame}[fragile]{Questions?}

Resources:

\begin{itemize}
\tightlist
\item
  Hilary Parker's
  \href{https://hilaryparker.com/2014/04/29/writing-an-r-package-from-scratch/}{Writing
  an R package from Scratch} blog post
\item
  Hadley Wickham's book \href{http://r-pkgs.had.co.nz/}{R Packages}
\item
  \texttt{\#rstats} on Twitter
\end{itemize}

@alexpghayes on Twitter\\
\href{mailto:alexpghayes@gmail.com}{\nolinkurl{alexpghayes@gmail.com}}

\textbf{Summer opportunity}: \texttt{MoMA} package with Michael Weylandt
and Dr.~Allen

\end{frame}

\end{document}
